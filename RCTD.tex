\section{RCTD}

Our ultimate goal is to determine the fractional contributions of each cell type to a particular sample. We do this by maximum likelihood, using the following hierarchical model:
\begin{align*}
    \mathsf{Y}_{i, j} \mid \lambda_{i, j}
    &\sim \text{Poisson}(N_i \lambda_{i, j})
    \\
    \log \lambda_{i, j} &= \log (\vec{\beta}_i \cdot \vec{\mu}_{j}) + \alpha_i + \gamma_j + \varepsilon_{i, j},
\end{align*}
where
\begin{itemize}
    \ii $\mathsf{Y}_{i, j}$
    is the random variable corresponding to the observed expression of gene $j$ at pixel $i$,
    \ii $N_i$ is the number of transcripts for pixel $i$,
    \ii $\vec{\beta}_i$ is the $K$-dimensional row vector of contributions from each cell type (where $K$ is the number of cell types in question) at pixel $i$,
    \ii $\vec{\mu}_j$ is the $K$-dimensional column vector of mean expressions of gene $j$ for each cell type,
    \ii $\alpha_i$ is a fixed pixel-specific effect.
    \ii $\gamma_j$ and $\varepsilon_{i, j}$ are random effects that introduce noise. $\gamma_j$ in particular is intended to account for platform effects that may over- or underrepresent certain genes. We let these be normally distributed with mean 0 and variance $\sigma_\gamma$, $\sigma_\varepsilon$ respectively.
\end{itemize}
Therefore, determining the fractional contributions of each cell type reduces to finding the maximum likelihood parameter $\vec{\beta}_i$ for each $i$.
\begin{ques}
    Now we have a ton of parameters, potentially thousands. $\beta$ alone introduces $K\times J$ of them. How do we do any useful estimation here?
\end{ques}
We proceed in the following steps:
\begin{enumerate}
    \ii 
    \vocab{Supervised estimation of cell type profiles}.
    
    Using a reference dataset, we estimate the parameters $\mu_j$ of expression levels for gene $j$, giving $\hat{\vec{\mu}}_j$ which will be used in the next steps.

    We can do this by obtaining a (e.g. scRNA-seq) reference annotated with cell types, after which $\vec{\mu}_j$ can be estimated as the empirical average normalized expression of gene $j$ within each cell type.
    \ii
    \vocab{Gene filtering}.

    Using the estimated expression profiles $\hat{\vec{\mu}}_j$, we filter out genes that are not highly variable across cell types.

    We can do this by taking the expression profiles $\hat{\vec{\mu}}_j$ and selecting genes with a minimum average expression and sufficiently high variance.\todo{does this discard genes that are, say, only expressed in one cell type? (average might be low, but gene could be good marker)}
    \ii
    \vocab{Platform Effect Normalization}.

    With an estimate for $\vec{\mu}_j$, it turns out we now have a way to estimate the platform effects $\gamma_j$ as well.
    The idea is that we can consider the average observed expression across pixels
    \[
    \mathsf{M}_j = \frac{1}{I} \sum\limits_{i=1}^{I} \mathsf{Y}_{i, j},
    \]
    whence
    \begin{align*}
        \log \EE_{\mathsf{M}_j | \vec{\lambda}_j}\left[ M_j \midd  \lambda_{1, j}, \ldots, \lambda_{I, j} \right]
        &= \log \left( \frac{1}{I} \sum\limits_{i=1}^{I} N_i \lambda_{i, j} \right)\\
        &= \log \left( \frac{1}{I} \sum\limits_{i=1}^{I} N_i \exp \left( \log (\vec{\beta}_i \cdot \vec{\mu}_j) +\alpha_i + \gamma_j + \varepsilon_{i, j} \right) \right)\\
        &= \gamma_j + \log \left( \frac{1}{I} \sum\limits_{i=1}^{I} N_i (\vec{\beta}_i \cdot \vec{\mu}_j)\exp \left( \alpha_i + \varepsilon_{i, j} \right) \right)\\
        &= \gamma_j + \log \left( \frac{1}{I} \sum\limits_{i=1}^{I} \left( \sum\limits_{k=1}^{K} \beta_{i, k}\cdot \mu_{k, j} \right)N_i \exp \left( \alpha_i + \varepsilon_{i, j} \right) \right)\\
        &= \gamma_j + \log \left( \sum\limits_{k=1}^{K} \mu_{k, j} \sum\limits_{i=1}^{I} \frac{N_i}{I} \beta_{i, k}\exp \left( \alpha_i + \varepsilon_{i, j} \right)  \right)\\
        &= \gamma_j + \log \left( \ol{N} \sum\limits_{k=1}^{K} \mu_{k, j} \left( \frac{1}{I}\sum\limits_{i=1}^{I} \frac{N_i}{\ol{N}} \beta_{i, k}\exp \left( \alpha_i + \varepsilon_{i, j} \right) \right)  \right)\\
        &= \gamma_j + \log \left( \ol{N} \sum\limits_{k=1}^{K} \mu_{k, j} B_{k, j}  \right),
    \end{align*}
    where
    \begin{align*}
        \ol{N}  \defeq \frac{1}{I} \sum\limits_{i=1}^{I} N_i \quad \text{and} \quad
        B_{k, j} \defeq \frac{1}{I} \sum\limits_{i=1}^{I} \frac{N_i}{\ol{N}} \beta_{i, k} \exp(\alpha_i+\varepsilon_{i, j}).
    \end{align*}
    $\mathsf{B}_{k, j}$ has some desirable properties. In particular, we have
    \begin{align*}
        \EE_{\hat{p}_{\mathsf{B}_{k, j}}}\left[B_{k, j}\right]
        &= 
        \EE_{\hat{p}_{\mathsf{N}, \varepsilon_{j}, \beta_k, \alpha}}\left[ \frac{N}{\EE_i[N]}\beta_k \exp(\alpha+\varepsilon_j) \right]
        \\
        &= \EE_{\hat{p}_{\varepsilon_{j}, \beta_k, \alpha}}\left[ \beta_k \exp(\alpha+\varepsilon_j) \right]
        \\
        &= \ol{\beta}_k\EE_{\hat{p}_{\varepsilon_{j}, \alpha}}\left[ \exp(\alpha+\varepsilon_j) \right]
        \\
        &\defeq \ol{\beta}_k\beta_0,
    \end{align*}
    and as $I\to \infty$, $\Var[B_{k, j}] \to 0$.\todo{Why is this the case? Central Limit Theorem with weak dependence or something?}  Therefore, as $I \to \infty$ we get $B_{k, j} \approx \ol{\beta}_k \beta_0$ so
    \[
    \log \EE_{\mathsf{M}_j | \vec{\lambda}_j}\left[ M_j \midd  \lambda_{1, j}, \ldots, \lambda_{I, j} \right]
    \approx \gamma_j + \log \beta_0 + \log \left( \ol{N} \sum\limits_{k=1}^{K} \mu_{k, j}\ol{\beta}_k \right).
    \]
    Now we can estimate the platform effects $\gamma_j$ by plugging in the MLE for $\beta_0$, $\ol{\vec{b}}$, and $\sigma_\gamma$.\todo{How do we go from the above equation to the MLE for the relevant parameters? Is it part of the EM algorithm?}
    \ii \vocab{Robust Cell Type Decomposition}.

    With $\hat{\mu}_{k, j}$ and $\hat{\gamma}_j$ determined, we then find the MLE estimate for $\alpha_i, \vec{\beta}_i$ and $\sigma_\varepsilon$ in our original model.
\end{enumerate}<++>
